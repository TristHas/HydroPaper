\documentclass{article}

\usepackage{arxiv}

\usepackage[utf8]{inputenc} % allow utf-8 input
\usepackage[T1]{fontenc}    % use 8-bit T1 fonts
\usepackage{hyperref}       % hyperlinks
\usepackage{url}            % simple URL typesetting
\usepackage{booktabs}       % professional-quality tables
\usepackage{amsfonts}       % blackboard math symbols
\usepackage{nicefrac}       % compact symbols for 1/2, etc.
\usepackage{microtype}      % microtypography
\usepackage{lipsum}		% Can be removed after putting your text content
\usepackage{graphicx}
\usepackage{natbib}
\usepackage{doi}

\title{
	A feasibility study of upper-layer reservoir operation for hydro power generation in Japanese dams
}

%\date{September 9, 1985}	% Here you can change the date presented in the paper title
%\date{} 					% Or removing it

\author{ Tristan \\
	Kobe University\\
	\texttt{tristan@people.kobe-u.ac.jp} \\
	\And
	Yoshimi \\
	Kobe University\\
	\texttt{yoshimi@kobe} \\
	\And
	Rousslan \\
	Kobe University\\
	\texttt{rousslan@kobe} \\
	\And
	Victor \\
	CNES\\
	\texttt{victor@cnes} \\	
	\And
	Takiguchi \\
	Kobe University\\
	\texttt{Takiguchi@kobe} \\	
	\And
	Oishi \\
	RIKEN\\
	\texttt{oishi@riken} \\	
}

% Uncomment to remove the date
\date{}

% Uncomment to override  the `A preprint' in the header
%\renewcommand{\headeright}{Technical Report}
%\renewcommand{\undertitle}{Technical Report}
%\renewcommand{\shorttitle}{\textit{arXiv} Template}
\renewcommand{\shorttitle}{ }

%%% Add PDF metadata to help others organize their library
%%% Once the PDF is generated, you can check the metadata with
%%% $ pdfinfo template.pdf
\hypersetup{
pdftitle={A template for the arxiv style},
pdfsubject={q-bio.NC, q-bio.QM},
pdfauthor={David S.~Hippocampus, Elias D.~Striatum},
pdfkeywords={First keyword, Second keyword, More},
}

\begin{document}
\maketitle

\begin{abstract}

% Optimal dam management needs good forecast of futur income

% Two main sources of uncertainty 

% Traditionally, the uncertainty has lead to conservative policies
Traditionally, the uncertainty in reservoir inflow forecast
have lead dam operators to adopt conservative operation strategies.
However two factors come to challenge this status quo:
First, as both fossile fuel and atomic plant receive increased criticism for
social and environmental xxx, 

% However two factors come to challenge this status quo:

% 1 As fire and atomic sources goes off, the value of low-carbon non-intermitent energy increases

% 2 As climate modeling and forecast technologies keep improving

% The risk and merit trade off may tilt the balance towards adopting more optimized dam operation policies.  

% This work aims to lay out the foundations for such proposal.

% To do so, we have assembled a dataset, proposed a conceptual framework, and analyze existing models.

% The goal is to quantify the impact of weather and hydrological forecast accuracies on hydro-power generation of Japanese national dams.

\end{abstract}

% keywords can be removed

\keywords{Dam Reservoir Operation \and Hydro Power Generation \and Discharge Forecast \and Reinforcement Learning \and Machine Learning}

\section{Introduction}
\label{sec:Introduction}

% 
Japanese power generation situation.

% 
Summary of contributions: 

\begin{itemize}
	\item Data operational
	\item Operational models for reservoir input flow
	\item Methodolody: Vertical integration.
	\item Open-source everything with easy-to-use tutorials for researchers to join.
\end{itemize}

Limitations

\begin{itemize}
	\item No multi-purpose
	\item No multi-dam
\end{itemize}

Nevertheless, we hope it's useful.


\section{Related Work}
\label{sec:Related Work}

Dam operation.
Discharge prediction.

\section{Methodolody}
\label{sec:Methodolody}

Show Figure.

\subsection{Overview}
\label{sec:Overview}

\subsection{Meteorological model}
\label{sec:Meteorological model}

\subsection{Hydrological Model}
\label{Hydrological Model}

\subsection{Dam Model}
\label{Dam Model}

\subsection{Grid Response}
\label{Grid Response}

The grid response model defines the reward $r$ obtained for a given power production $P$ at a given time $t$: 

$$ r = f(P, t)$$

This reward can be though of as a profit in a free-market economy, and is the sum of two terms:

$$r = r_n + r_p$$

The positive quantifies the benefit of energy produced.
Currently, we use $f(x)=x$, considering constant value over time of the generated power.
However, hydro's value lies in its non-intermittence to supplement intermittent sources.
It would be interested to couple this with actual demand response simulations in the future.

The negative price quantifies the cost of a dam failure.
We considered this price to be the sum over 
This means that the negative impact of a single dam failure over the whole period of activity 
would outweight the benefits of full power production of the whole dam operation.

\subsection{Dam Operation Model}
\label{sec: Dam Operation Model}



\section{Proposed Dataset}
\label{sec:Proposed Dataset}


\section{Optimal Dam Operation}
\label{sec:Hydrological Uncertainty}

We start by optimizing the dam without any uncertainty.

\subsection{Baseline}
\label{sec:Dam Operation baseline}

\subsection{Optimal Operator}
\label{sec:Optimal Operator}

\subsection{Operation Under Uncertainty}
\label{sec:Operation Under Uncertainty}

\subsection{Experiment}
\label{sec:Hydrological Uncertainty}

 - Answers the following question:

1. What is the forecast horizon needed in order to keep the dam from overflowing?
-> Show as input distribution.

2. Impact of state design on RL baseline.

3. Show the ideal model is brittle. How?

\section{Hydrological Uncertainty}
\label{sec:Hydrological Uncertainty}

In this section, we use assimilated data, no forecast.

\subsection{Hydrological Models}
\label{sec:Hydrological Models}

Use Camaflood, conceptual rainfall runoff, linear, non-linear and deep machine learning models.

\subsection{Experiments}
\label{sec:Hydrological Experiments}

\subsubsection{Discharge prediction}

Show accuracy of different models.

\subsubsection{Dam operation}

Show reward of different models.

\begin{table}
	\caption{Model accuracy}
	\centering
	\begin{tabular}{lll}
		\toprule
		\multicolumn{2}{c}{Part}                   \\
		\cmidrule(r){1-2}
		Name     & Description     & Size ($\mu$m) \\
		\midrule
		Dendrite & Input terminal  & $\sim$100     \\
		Axon     & Output terminal & $\sim$10      \\
		Soma     & Cell body       & up to $10^6$  \\
		\bottomrule
	\end{tabular}
	\label{tab:table}
\end{table}

\section{Meteorological Uncertainty}
\label{sec:Meteorological Uncertainty}

\subsection{Meteo Data}
\label{sec:Meteo Data}

\subsection{Experiments}
\label{sec:Meteo Experiments}

Use different data source.


\subsection{Meteorological Uncertainty}

\section{Limitations}

\begin{itemize}
	\item No multi-purpose
	\item No multi-dam
 	\item No explicit evaporation modeling
	\item Heuristic dimensioning.
	\item No river and snow gauge.  
\end{itemize}

\section{Conclusion}
\label{sec:Conclusion}

Conclusion

\bibliographystyle{unsrtnat}
\bibliography{references}  %%% Uncomment this line and comment out the ``thebibliography'' section below to use the external .bib file (using bibtex) .


%%% Uncomment this section and comment out the \bibliography{references} line above to use inline references.
% \begin{thebibliography}{1}

% 	\bibitem{kour2014real}
% 	George Kour and Raid Saabne.
% 	\newblock Real-time segmentation of on-line handwritten arabic script.
% 	\newblock In {\em Frontiers in Handwriting Recognition (ICFHR), 2014 14th
% 			International Conference on}, pages 417--422. IEEE, 2014.

% 	\bibitem{kour2014fast}
% 	George Kour and Raid Saabne.
% 	\newblock Fast classification of handwritten on-line arabic characters.
% 	\newblock In {\em Soft Computing and Pattern Recognition (SoCPaR), 2014 6th
% 			International Conference of}, pages 312--318. IEEE, 2014.

% 	\bibitem{hadash2018estimate}
% 	Guy Hadash, Einat Kermany, Boaz Carmeli, Ofer Lavi, George Kour, and Alon
% 	Jacovi.
% 	\newblock Estimate and replace: A novel approach to integrating deep neural
% 	networks with existing applications.
% 	\newblock {\em arXiv preprint arXiv:1804.09028}, 2018.

% \end{thebibliography}


\end{document}
