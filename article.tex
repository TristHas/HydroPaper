\documentclass{article}

\usepackage{arxiv}

\usepackage[utf8]{inputenc} % allow utf-8 input
\usepackage[T1]{fontenc}    % use 8-bit T1 fonts
\usepackage{hyperref}       % hyperlinks
\usepackage{url}            % simple URL typesetting
\usepackage{booktabs}       % professional-quality tables
\usepackage{amsfonts}       % blackboard math symbols
\usepackage{nicefrac}       % compact symbols for 1/2, etc.
\usepackage{microtype}      % microtypography
\usepackage{lipsum}		% Can be removed after putting your text content
\usepackage{graphicx}
\usepackage{natbib}
\usepackage{doi}

\title{
	% An investigation into
	On the feasibility of leveraging upper storage zones
  for hydro-power generation in Japanese public dams operation
}

%\date{September 9, 1985}	% Here you can change the date presented in the paper title
%\date{} 					% Or removing it

\author{ Tristan \\
	Kobe University\\
	\texttt{tristan@people.kobe-u.ac.jp} \\
	\And
	Yoshimi \\
	Kobe University\\
	\texttt{yoshimi@kobe} \\
	\And
	Rousslan \\
	Kobe University\\
	\texttt{rousslan@kobe} \\
	\And
	Victor \\
	CNES\\
	\texttt{victor@cnes} \\
	\And
	Takiguchi \\
	Kobe University\\
	\texttt{Takiguchi@kobe} \\
	\And
	Oishi \\
	RIKEN\\
	\texttt{oishi@riken} \\
}

% Uncomment to remove the date
\date{}

% Uncomment to override  the `A preprint' in the header
%\renewcommand{\headeright}{Technical Report}
%\renewcommand{\undertitle}{Technical Report}
%\renewcommand{\shorttitle}{\textit{arXiv} Template}
\renewcommand{\shorttitle}{ }

%%% Add PDF metadata to help others organize their library
%%% Once the PDF is generated, you can check the metadata with
%%% $ pdfinfo template.pdf
\hypersetup{
pdftitle={A template for the arxiv style},
pdfsubject={q-bio.NC, q-bio.QM},
pdfauthor={David S.~Hippocampus, Elias D.~Striatum},
pdfkeywords={First keyword, Second keyword, More},
}

\begin{document}
\maketitle

\begin{abstract}

% Optimal dam management needs good forecast of future income
Dam operation is a problem of control under uncertainty,
in which an optimal operation policy is sought for so as to maximize
a reward function given uncertain forecasts of river discharges flowing into the dam reservoir.
%as optimal policies are functions of
%noisy forecasts of river discharges incoming to dam reservoirs.
% Two main sources of uncertainty
The uncertainty in discharge forecast can be attributed to two main factors:
Uncertainty in atmospheric (mostly precipitation) forecasts,
and in hydrological (interception, evaporation and runoff) modeling.
% Traditionally, the uncertainty has lead to conservative policies
High levels of uncertainty and abundant alternative sources of power generation
have lead Japanese public dam operators to adopt conservative operation strategies
so as to minimize the risk of dam failures.
However two factors may come to challenge this status quo:
% 1 As fire and atomic sources goes off, the value of low-carbon non-intermitent energy increases
%First, as social and environmental pressures threaten the long term viability of fossile fuel
%and atomic plant receive increased criticism for social and environmental reasons.
First, social and environmental pressures on fossil fuel and nuclear energy production
are foreseen to increase the value of hydrological power generation.
% 2 As climate modeling and forecast technologies keep improving
Second, advances in both environmental modeling and statistical inference models
are foreseen to increase the accuracy of forecast.
Combined, these two factors may challenge the current risk-benefit analysis
of dam operation towards policies more focused on energy production.
This work aims to lay out the foundations for a large scale study
on the feasibility of implementing such policies.
We do so by providing the following three contributions.
First we release a dataset of Japanese public dam allowing to benchmark the accuracy of
various hydrological models coupled with different atmospheric forecasts.
Second, we propose a conceptual framework to analyse the errors in discharge forecast
introduced by atmospheric and hydrological models, as well as their impact on dam operation.
Finally we analyse a variety of existing hydrological and atmospheric forecasts,
and provide insight regarding the key challenges to implement hydro-power optimized
dam operation procedure.
%We have open-sources all data and code,
%and provide interactive tutorials for data scientists, hydrologist or climate specialist to easily join in on our effort.
%assembled a dataset, proposed a conceptual framework,
%and analyze the accuracy  models.
%The goal is to quantify the impact of weather and hydrological
%forecast accuracies on hydro-power generation of Japanese national dams.
% Keyword: integrated approach.

\end{abstract}

% keywords can be removed

\keywords{Dam Reservoir Operation \and Hydro Power Generation \and Discharge Forecast \and Reinforcement Learning \and Machine Learning}

\section{Introduction}
\label{sec:Introduction}

%% Many mountains and a lot of rain=good for hydro power.
A mountainous topology and a heavy rain climate lend Japan a high potential for hydro power generation \cite{}.
% Historical perspective
Historically\cite{}, Japan has extensively relied on its hydrological power generation up until the second world war.
As the period of great economic development called for increased energy consumption,
fossil fuel plants have come to fill this demand for their ability to quickly and efficiently provide for this sudden jump in demand.
Later, the oil shock has seen Japan develop nuclear power generation for strategical reasons, to ensure its energy independency.

% Current situation
%% Nuclear and fossil fuel contested
Nuclear incidents and international pledges to reduce carbon emissions have come to threaten the long term viability of the current energy mix.

%% Renewable energies are intermittent
As solar and wind power generation is being pushed forward,
they are widely considered as insufficient in their own due to their intermittent nature. \cite{}

%% Variety of hydro provide low-carbon, non-intermittent, both base load and demand response.
Hydro is low-carbon and a variety of hydro power plant can bring both base load and demand response.

%% For all its benefits, voices are calling for the optimization of Japanese dam operations.
For all its benefits, voices are calling for the optimization of Japanese dam operations. \cite{}
%% Probably more to say, what exactly?

% Fundamental problem:
The fundamental problem of dam operation is that of two opposing objectives\cite{}:
On the one hand energy production (as well as irrigation)
benefits from keeping high levels of water into the reservoir.
Indeed the higher water levels are, the more potential energy it stores,
thus the more energy it can create for a given volume of output water discharge.
On the other hand, flood control benefits from low level of reservoir filling,
as lower water levels enable to hold more incoming water,
thus better buffering high discharge flows to prevent potential floods.
An optimal operation policy is one that maximizes energy production while minimizing the risk of flooding.
However, keeping appropriate levels in the reservoir at each time requires knowledge
of the amount of water flowing into the reservoir ahead of time, i.e. accurate river discharge forecast.
% In particular dam failures.
In particular, optimizing dam operation for energy production comes with a risk of dam failures:
Dam failure happens when the the dam reaches its full capacity due to large volume
of water flowing into the reservoir.
In such cases, the reservoir is then brutally emptied to prevent the dam infrastructure to collapse
under the weight of the water, thus flooding the downstream areas
at great cost for the population.

% In the absence of accurate forecast, reservoir zoning is used.
In the absence of reliable discharge forecast,
reservoir zoning operation strategies are often employed.
Reservoir zoning, as illustrated in Figure 1,
consists in allocating different purposes to different layers of the reservoir.
The xxx layer defines a lower bound on water level under which the water accumulated without release.
The xx layer is xxx.
TODO

% Illustration of the impact of forecast accuracy.
Beyond reservoir zoning operation, accurate river discharge forecasts would enable dam operators
to keep reservoirs high beyond the safety threshold of the lower xxx zone,
in order to maximize energy production
while ensuring dam failures do not occur.
In contrast, inaccurate river discharge forecast would force dam operators
to keep water levels below the safety threshold of the
in the reservoir in order to avoid dam failures in case of unexpected sudden inflows,
resulting in less efficient power generation policies.
The dam power generation efficiency is thus inherently limited by the accuracy of the input river discharge.

% Nature of uncertainty
There are two main sources of errors in river discharge forecast:
uncertainty in the atmospheric forecast and uncertainty in the hydrological modeling.
In this paper, we quantify each the impact of each of these errors
on the dam reservoirs's input discharge accuracy
and on the power generation policy's efficiency.
Our analysis will lead us to answer questions such as follows:

\begin{itemize}
  \item What is the relationship between river discharge forecast accuracy and dam operation efficiency?
	\item Are existing atmospheric forecast and hydrological models accurate enough to
	safely enable the deployment of policies maximizing hydro
	power generation beyond fixed reservoir zoning operations?
	Which existing models are most accurate?
	\item Is atmospheric forecast uncertainty more impactful
	than hydrological uncertainty on the inflow river discharge accuracy?
	Improvement on which of these models would be most beneficial to dam operation?
	%\item What is the impact of the accuracy of the atmospheric forecast versus the hydrological modeling?
	\item What is the impact of factors such as snow melt, latitude, seasonality, etc.,
	on discharge forecast accuracy and operation efficiency?
\end{itemize}

% Summary of contributions
In addition to the above questions, this paper provides the following contributions:

\begin{itemize}
	\item A dataset of historical hourly input river discharge as well and
	atmospheric variables (both forecasted and reanalyzed) projected onto their
	respective drainage area for xxx Japanese public dam over,
	and a methodology to benchmark predictive models.
	\item Near real-time operational reservoir input discharge forecasting for
	each of these dams.
	\item Open-source project, with tutorial to help data anlysts and modelist join our effort efficiently.
\end{itemize}

% Notes on limitations
Our work, in its current form, suffer important limitations.
These limitations are discussed at length in Section XXX,
but the most notable ones deserve a mention at this point in the discussion:

\begin{itemize}
	\item Our proposed dam operation policies are not operational:
	we only consider a unique objective (i.e. energy production vs. flood prevention),
	while most dams serve several purpose, which impacts their operation policies.
	This is an important limitation which deserves nuanced discussion,
	which we provide in Section XXX.
	\item We do not consider multi-dam operation settings.
	Several Japanese public dams are located downstream of other dams,
	so the upstream dam operation affects the river input discharge of the downstream dam,
	and the operation of both dams need some synchronization for optimal operation.
	This is a well-studied setting \cite{} which, for the sake of simplicity,
	we do not consider in our analysis.
	\item
\end{itemize}

% Positive note
Nevertheless, and despite these limitations,
we believe our analysis brings insightful contributions
and invite interested reader to follow along.

\section{Related Work}
\label{sec:Related Work}

% Discharge computation
Hydro modeling

% Discharge forecast
Hydro forecast

% Dam operation
Dam operation.

% What else?
What else?

\section{Methodolody}
\label{sec:Methodolody}

This sections ...

\subsection{Overview}
\label{sec:Overview}

% What else
Figure 1 ...


\begin{figure}[t]
  \centering
  \includegraphics[width=.5\linewidth]{Figures/Overview1} %height=3\baselineskip
  \caption[First figure]{Illustration of the proposed approach}
\end{figure}



\subsection{Meteorological model}
\label{sec:Meteorological model}

The meteorological model provides atmospheric variables.


\subsection{Hydrological Model}
\label{Hydrological Model}

The hydro model output river discharge incoming to the dam reservoir
from the atmospheric variables, either observed or forecasted.

\subsection{Dam Model}
\label{Dam Model}

Reservoir geometry: We infer the maximum volume from the observations.
We infer the maximum height from Wikipedia. We then implemented both trapezoid and truncated cone geometry for the dam reservoir.

Operation: We define two doors, one big door that does not produce energy.
The dimension of the door is inferred from data (= to the max observed outflow release).
One small door that produces electricity (heuristic size = 1/50th of the big door).


\subsection{Grid Response}
\label{Grid Response}

The grid response model defines the reward $r$ obtained for a given power production $P$ at a given time $t$.
This reward can be though of as a profit in a free-market economy, and is the sum of two terms:

$$ r = r_p(P, t) + r_n(t)$$

The positive reward quantifies the benefit of energy produced.
Currently, we use $r_p(P, t)=P$, considering constant value over time of the generated power.
However, hydro's value lies in its non-intermittence to supplement intermittent sources.
Integrating the fluctuations of intermitent sources and their impact on the value of hydro power generation
at different seasons and time of the day may prove interesting for future research.

The negative reward quantifies the cost of a dam failure.
We set this cost to be constant $r_n(t)=C$ per dam,
as the maximum positive reward that can be produced by this dam over the whole period of operation,
so that the negative impact of a single dam failure over the whole period of activity
would outweight the benefits of full power production of the whole dam operation.

\subsection{Dam Operation Model}
\label{sec: Dam Operation Model}

Every hour, the dam operator can either open the big door, the small door or do nothing for one hour:
Action space $ \mathcal{A} = \{a1, a2, a3 \}$

Decisions are made based on the current level of the reservoir as well as the future input discharge forecast for a given horizon.
State space S = [Current volume / height, inflow forecasts]

Our goal is to learn a policy function $\Pi: \mathcal{S} \rightarrow \mathcal{A}$ that maximizes reward.
The reward for electricity production is proportional to the height of the water level times the volume of water discharged. Negative reward is also associated with dam failures:


\section{Proposed Dataset}
\label{sec:Proposed Dataset}


\section{Optimal Dam Operation}
\label{sec:Hydrological Uncertainty}

We start by optimizing the dam without any uncertainty.

\subsection{Baseline}
\label{sec:Dam Operation baseline}

\subsection{Optimal Operator}
\label{sec:Optimal Operator}

\subsection{Operation Under Uncertainty}
\label{sec:Operation Under Uncertainty}

\subsection{Experiment}
\label{sec:Hydrological Uncertainty}

 - Answers the following question:

1. What is the forecast horizon needed in order to keep the dam from overflowing?
-> Show as input distribution.

2. Impact of state design on RL baseline.

3. Show the ideal model is brittle. How?

\section{Hydrological Uncertainty}
\label{sec:Hydrological Uncertainty}

In this section, we use assimilated data, no forecast.

\subsection{Hydrological Models}
\label{sec:Hydrological Models}

Use Camaflood, conceptual rainfall runoff, linear, non-linear and deep machine learning models.

\subsection{Experiments}
\label{sec:Hydrological Experiments}

\subsubsection{Discharge prediction}

Show accuracy of different models.

\subsubsection{Dam operation}

Show reward of different models.

\begin{table}
	\caption{Model accuracy}
	\centering
	\begin{tabular}{lll}
		\toprule
		\multicolumn{2}{c}{Part}                   \\
		\cmidrule(r){1-2}
		Name     & Description     & Size ($\mu$m) \\
		\midrule
		Dendrite & Input terminal  & $\sim$100     \\
		Axon     & Output terminal & $\sim$10      \\
		Soma     & Cell body       & up to $10^6$  \\
		\bottomrule
	\end{tabular}
	\label{tab:table}
\end{table}

\section{Meteorological Uncertainty}
\label{sec:Meteorological Uncertainty}

\subsection{Meteo Data}
\label{sec:Meteo Data}

\subsection{Experiments}
\label{sec:Meteo Experiments}

Use different data source.


\subsection{Meteorological Uncertainty}

\section{Limitations}

\begin{itemize}
	\item No multi-purpose
	\item No multi-dam
 	\item No explicit evaporation modeling
	\item Heuristic dimensioning.
	\item No river and snow gauge.
\end{itemize}

\section{Conclusion}
\label{sec:Conclusion}

Conclusion

\bibliographystyle{unsrtnat}
\bibliography{references}  %%% Uncomment this line and comment out the ``thebibliography'' section below to use the external .bib file (using bibtex) .

\end{document}
